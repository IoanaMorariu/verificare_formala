Această lucrare a evidențiat importanța folosirii Rețelelor Neuronale Generative Adversariale Condiționate (cGAN) în recunoașterea de imagini. Au fost oferite mai multe detalii legate de benchmark-ul cGAN, instalarea și rularea instrumentelor alpha-beta-CROWN și NeuralSAT și interpretarea rezultatelor obținute.

S-a analizat performanța și funcționarea rețelelor neuronale prin folosirea a două tool-uri importante: alpha-beta-CROWN și NeuralSAT. 

Instrumentele folosite au avut rol în verificarea formală a corectitudinii rețelelor neuronale. S-a evaluat capacitatea rețelelor de a se alinia cu condițiile de intrare specificate.

Rezultatele în urmă rulării, au fost puse manual în tabele pentru a faciliza compararea cu competiția. În compararea rezultatelor noastre cu cele din competiție, s-a observat o aliniere în ceea ce privește rezultatele satisfăcătoare sau nesatisfăcătoare. Diferențe au existat în schimb, la timpii necesari pentru verificare.

În concluzie, lucrarea a subliniat eficiența utilizării cGAN în contextul recunoașterii imaginilor. A analizat instrumentele de verificare neuronală și a evidențiat cât de importante sunt în dezvoltarea tehnologiilor de inteligență artificială.
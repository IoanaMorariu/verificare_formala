\hspace{0.5 cm}
În această lucrare, am analizat în detaliu benchmark-ul cGAN din cadrul competiției VNN-Comp2023. Am descris pașii de instalare și rulare a instrumentelor alpha-beta-CROWN și NeuralSAT. Ulterior, am interpretarea rezultatele obținute în urma rulării. Aceste acțiuni au avut ca scop analizarea performanței și funcționării rețelelor neuronale.

Instrumentele folosite au fost dezvoltate special pentru verificarea formală a corectitudinii rețelelor neuronale. Prin intermediul rulării tool-urilor pe setul cGAN, s-a evaluat capacitatea rețelelor, din cadrul benchmark-ului, de a se alinia cu condițiile de intrare specificate în fișierele .vnnlib.

Rezultatele în urmă rulării, au fost extrase în tabele pentru a facilita compararea cu rezultatele din cadrul competiției. În compararea rezultatelor noastre cu cele din competiție, s-a observat o aliniere în ceea ce privește rezultatele satisfiabile și nesatisfiabile. Diferențe au existat, în schimb, la timpii necesari pentru verificare. Spre final, am menționat despre eficiența utilizării cGAN în contextul recunoașterii imaginilor.
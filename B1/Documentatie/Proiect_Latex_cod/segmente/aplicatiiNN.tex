\hspace{0.5 cm}
Verificarea rețelelor neuronale este o etapă crucială în dezvoltarea și aplicarea modelelor avansate. Aceasta este esențială din mai multe motive importante.

În primul rând, corectitudinea și fiabilitatea rețelelor neuronale sunt imperative. Ele sunt utilizate într-o varietate de domenii, de la medicină și tehnologie până la securitate cibernetică și vehicule autonome. Verificarea ne asigură că aceste rețele funcționează conform așteptărilor, oferind rezultate precise și fiabile într-o gamă largă de situații.

Siguranța reprezintă un alt aspect esențial. În domenii critice precum medicina sau industria automotive, erorile în funcționarea rețelelor neuronale pot avea consecințe grave. Verificarea este necesară pentru a identifica și remedia eventualele vulnerabilități care ar putea pune în pericol sistemul.

În industria auto, utilizarea învățării profunde și a viziunii artificiale pentru generarea de imagini noi joacă un rol esențial. Aceste tehnologii permit generarea de imagini sintetice pentru antrenarea și testarea vehiculelor autonome, simularea diverselor scenarii de conducere și optimizarea sistemelor de vizualizare și senzorilor \cite{googleAppliedDeep}. Ele contribuie la îmbunătățirea siguranței și eficienței în transporturi și la dezvoltarea vehiculelor autonome mai avansate. Utilizarea rețelelor neuronale în generarea de imagini pentru industria auto contribuie prin antrenarea eficientă a algoritmilor, simularea sigură a situațiilor de trafic și optimizarea senzorilor, accelerând dezvoltarea și îmbunătățirea vehiculelor autonome.

De asemenea, verificarea rețelelor neuronale contribuie la prevenirea bias-ului și discriminării. Aceste rețele pot fi influențate de prejudecăți încorporate în datele de antrenament. Prin teste și evaluări riguroase, putem identifica și corecta aceste bias-uri pentru a asigura obiectivitate și corectitudine în rezultatele obținute.
\hspace{0.5 cm} Verificarea rețelelor neuronale este un proces esențial în dezvoltarea și implementarea acestor modele avansate. Această practică reprezintă un pilon fundamental din mai multe motive cheie.

În primul rând, corectitudinea și fiabilitatea rețelelor neuronale sunt imperative. Acestea sunt utilizate într-o varietate de domenii, de la medicină și tehnologie la securitate cibernetică și vehicule autonome. Verificarea asigură că aceste rețele operează conform așteptărilor, furnizând rezultate precise și de încredere într-o gamă largă de scenarii.

Siguranța este un alt aspect crucial. În aplicații critice, cum ar fi cele medicale sau cele legate de siguranța vehiculelor, erorile în funcționarea rețelelor neuronale pot avea consecințe grave. Verificarea acestora este vitală pentru a identifica și a remedia potențialele vulnerabilități care ar putea compromite siguranța sistemelor.

De asemenea, verificarea rețelelor neuronale ajută la prevenirea bias-ului și discriminării. Aceste rețele pot fi susceptibile la prejudecăți încorporate din datele de antrenament. Prin testare și evaluare riguroasă, se poate identifica și corecta aceste bias-uri pentru a asigura obiectivitate și echitate în rezultatele furnizate.